\section{零八}
\vspace*{2\ccwd}
\begin{adjustwidth}{2cm}{1cm}
    \Large\kaishu{
    2019年,12月13日,周五

    \vspace*{2\ccwd} 

    常时,每每审思分析自我的心理,行诸笔端总不过一怀悱恻,或为情苦。自己对自己讲道理,多少显得傻气,
    可而今我偏欲说将出来,明心见性,无畏浅浮。    

    \vspace*{\ccwd} 

    一曰:借他人之视观我,自卑若枉生,如何自信?

    答曰:至今仍未明真实兴味之所存,每行之事,皆事或须行,不成则油然懊恼,得成觉其应当,累忧寡欢,
    故而事临便生避退思,百无从怀自感一无所能,自卑之念缘起,渐而深;倘有喜好可尽心以付,欲力可沉,
    不嗟枉然,无谓结果,皆全意足。如此,心念有赖,行事有度,于是自信生。盼愿当真有我愿为,罔顾其他,
    神凝一也,力尽有的,敬事如,人亦如。

    \vspace*{\ccwd} 或曰:见近旁人之优越我,嫉妒何由?
    
    答曰:性也,此外更无知。
}
\end{adjustwidth}
\newpage