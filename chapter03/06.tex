\section{零六}
\vspace*{2\ccwd}
\begin{adjustwidth}{2cm}{1cm}
    \Large\kaishu{
    2019年,11月23日,周六

    \vspace*{2\ccwd} 

    \hspace*{2em}两人相处,其凭生烦恼之根由在于迁就习惯,且当遵彼习惯与本心相违,于是忧患滋衍。初时,二人因
    共同之喜好或目标走至一起,欣欢之甚而忽略那些微的异别,不以为意。只当彼此结合为爱侣,从兹事之琐杂使人应对无暇,
    若此时再执著于协同步调,依陈规为法则,自是矛盾突起。新人之成结乃为破立,破出原生之家,立建新生之家,其维行之道
    若此:抛掷旧生观念于先,再以演进的心态发展这幼弱的关系。以可作共识的根本原则为基石,于漫漫的历程中预见或遇见问题,
    在解决困难中拾补恰当的规条来搭建维系之框架。其间,莫倚赖从来已有的主观经验,当有靡无巨细的耐心,去创造适于两人的
    相处之法。须理解人会随环境、年岁和经历而变化心态,往往忽视中便让我们觉得某些似为突变,实则不然。激情或可消,但不
    该麻木。时时当以新角度掘寻对方可爱的地方,亦可促励爱人自我发展新的美,不应厌足。反观己,常能以新颖貌现于对方,在
    演变和惊喜中使感情得以维新,当费思量。

    \hspace*{2em}顿生此感,零乱记之。

    
}
\end{adjustwidth}
\newpage